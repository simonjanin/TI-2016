	\documentclass{article}
	\usepackage{blindtext}
	\usepackage[utf8]{inputenc}
	
	 \usepackage{amssymb}
	 
	 \usepackage{listings}
	 
	 \usepackage{xparse}

	
	\title{Theoretische Informatik - Übungsblatt 2}
	\author{Alexandre Roque (14-938-278)\\Simon Janin (12-814-760)}
	\date{\today}
\begin{document}
    \maketitle
    
    \section*{Aufgabe 4a}
    
    Wir geben zunächst für jedes n $\in \mathbb{N}$  ein Pascal-Programm an, das $w_{n}$ erzeugt:
    
    \lstset{language=Pascal}
    \begin{lstlisting}
    begin
    a := n;
    a := 2*a;
    
    b := 1;
    for i := 1 to a do
    	b := b*2;
    	
    c := 1;
    for i := 1 to b do 
    	c := c*2;
    	
    for i := 1 to c do
    	write(0);
    for i := 1 to c do
    	write(1);	 		
    end
    	
    \end{lstlisting}
    
    \par
    \noindent
    Wir schreiben das obige Pascal-Programm mithilfe der Potenz-Notation:     \lstset{language=Pascal}
    \begin{lstlisting}
    	
	begin
	a := 2n;
	b := 2^a;
	c := 2^b;
	for i := 1 to c do 
		write(0);
	for i := 1 to c do
		write(1);
	end  	
    	
    \end{lstlisting}
    
    \par
    
    Das Programm hängt nur von 2n ab, der Rest steht unter einer beliebigen Konstante c.
    Insgesamt, die Kolmogorov-Komplexität bezüglich n ist die folgende:
    
    \par
    
    \begin{center}
    	$K(w_{n}) \le \log_{2}(n)+ c$
    \end{center}
    
    \par
    
    Schauen wir jetzt die Länge von $w_{n}$. Da $w_{n} = 0^{2^{2^{2n}}}$ haben wir folgendes: 
    
    \begin{center}
    $|w_{n}| = 2^{2^{2n}}$ \\
    $\log_{2}(|w_{n}|) = 2^{2n}$ \\
    $\log_{2} \log_{2} |w_{n}|= 2n$ \\
    $\frac{1}{2} log_{2} \log_{2} |w_{n}|= n$
    \end{center}
    
    \par
	
	$\Rightarrow K(w_{n}) \le \log_{2}(\frac{1}{2} \log_{2} \log_{2} |w_{n}|) + c$ \\
	$\Rightarrow K(w_{n}) \le \log_{2}(\log_{2} \log_{2} |w_{n}|) - \log_{2}(2) + c$ \\
	$\Rightarrow K(w_{n}) \le \log_{2}(\log_{2} \log_{2} |w_{n}|) + c'$ wobei c' = c - $\log_{2}(2)$
	
	\section*{Aufgabe 4b}
	\begin{center}
	$K(y_{n}) \le \log_{2} \log_{2} \sqrt[3]{y_{n}} + c$ \\
	$\Rightarrow n =  \log_{2} \sqrt[3]{y_{n}}$ \\
	$\Leftrightarrow n = \frac{1}{3} \log_{2} y_{n} $ \\
	$\Leftrightarrow 3n = \log_{2} y_{n} $ \\
	$\Leftrightarrow 2^{3n} = y_{n} $
	\end{center}
	
	
	 
	\par 
	\noindent
	Von der Definition 2.18 des Buches wissen wir dass $K(y_{n}) = K(Bin(y_{n}))$ gilt. Somit definieren wir ein $a_{n} = Bin(y_{n}) = 1(0)^{2^{3n} - 1}$
    
    \par 
    \noindent
    Wir definieren die Zahlenfolge $y_{1}, y_{2}, y_{3}...$ durch $y_{n} = 2^{3n}$ für alle $ n \in \mathbb{N} - {0}$.
    Offenbar gilt $y_{n} < y_{n+1}$ und $n = \log_{2} \sqrt[3]{y_{n}}$ für alle $n$.
    
   
\end{document}
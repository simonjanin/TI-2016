	\documentclass{article}
	\usepackage{blindtext}
	\usepackage[utf8]{inputenc}
	
	 \usepackage{amssymb}
	 
	 \usepackage{listings}
	 
	 \usepackage{xparse}

	
	\title{Theoretische Informatik - Übungsblatt 2}
	\author{Alexandre Roque (14-938-278)\\Simon Janin (12-814-760)}
	\date{\today}
\begin{document}
    \maketitle
    

    \section*{Aufgabe 6}
    
    Wir definieren $A_{n}$, $\forall n \in \mathbb{N}$ als die Menge von allen natürlichen in dem Intervall:
    \begin{center}
    $[2^n, 2^{n+1} - 1]$, bzw. $A_{n} = \{x_{n}|$ $2^n \le x_{n} \le 2^{n+1} - 1, x_{n} \in \mathbb{N}\}$
    \end{center}
    
    
    \par 
    \noindent
    Wir wissen von der Aussage, dass die untere Schranke von jeder Menge $A_{n}$: 
	\begin{center}
	$2^n$ ist. \\
\end{center}	    
   
   	\noindent
    Jetzt möchten wir alle natürlichen Zahlen von diesen Mengen $A_{n}$ als binäre Kodierung repräsentieren. \\
	Erzeugen wir dann folgende Menge $B_{n}$:
	
	\begin{center}
	 $B_{n} = \{y_{n} |$ $y_{n} = Bin(x_{n})$ wobei $x_{n} \in A_{n}\}$
\end{center}	    
    
    
    \par 
  	\noindent
  	Dann alle wörter von $y_{n} \in B_{n}$ haben eine Länge von $n + 1$:
	\begin{center}
	$|y_{n}| = n + 1$, $\forall y_{n} \in B_{n}$
\end{center}	  	
  
    \par 
    \noindent
    Somit haben wir eine Kolmogorov-Komplexität von: 
    \begin{center}
    $|y_{n}| + c \ge K(y_{n})$
    \end{center}
    \par 
    \noindent
    Wir wissen, dass $K(x_{n}) = K(Bin(x_{n})) \ge n - i$. Dann es ergibt die folgende Gleichung was  die Aussage für beliebige Konstanten $c$ und $c'$ beweist: 
    \begin{center}
    $|y_{n}| + c = (n + 1) + c = n + c' \ge K(y_{n}) = K(Bin(x_{n})) = K(x_{n}) \ge n - i$, \\
    $\forall y_{n} \in B_{n}$, $x_{n} \in A_{n}$, $\forall i \in \mathbb{N}$ und $i < n$
    \end{center}
    
    \par 
    \noindent
    Insgesamt gilt es:
    \begin{center}
    $n + c' \ge K(x_{n}) \ge n - i$, \\
   	$\forall x_{n} \in A_{n}$, $\forall i \in \mathbb{N}$ und $i < n$
    \end{center}
\end{document}
	\documentclass{article}
	\usepackage{blindtext}
	\usepackage[utf8]{inputenc}
	
	 \usepackage{amssymb}
	 
	 \usepackage{listings}
	 
	 \usepackage{xparse}

	
	\title{Theoretische Informatik - Übungsblatt 2}
	\author{Alexandre Roque (14-938-278)}
	\date{\today}
\begin{document}
    \maketitle
    
    \section*{Aufgabe 4a}
    
    Wir geben zunächst für jedes n $\in \mathbb{N}$  ein Pascal-Programm an, das $w_{n}$ erzeugt:
    
    \lstset{language=Pascal}
    \begin{lstlisting}
    begin
    ax := n;
    a := 2*a;
    
    b := 1;
    for i := 1 to a do
    	b := b*2;
    	
    c := 1;
    for i := 1 to b do 
    	c := c*2;
    	
    for i := 1 to c do
    	write(0);
    for i := 1 to c do
    	write(1);	 		
    end
    	
    \end{lstlisting}
    
    \par
    \noindent
    Wir schreiben das obige Pascal-Programm mithilfe der Potenz-Notation:     \lstset{language=Pascal}
    \begin{lstlisting}
    	
	begin
	a := 2n;
	b := 2^a;
	c := 2^b;
	for i := 1 to c do 
		write(0);
	for i := 1 to c do
		write(1);
	end  	
    	
    \end{lstlisting}
    
    \par
    
    Das Programm hängt nur von 2n ab, der Rest steht unter einer beliebigen Konstante c.
    Insgesamt, die Kolmogorov-Komplexität bezüglich n ist die folgende:
    
    \par
    
    \begin{center}
    	$K(w_{n}) \le \log_{2}(n)+ c$
    \end{center}
    
    \par
    
    Schauen wir jetzt die Länge von $w_{n}$. Da $w_{n} = 0^{2^{2^{2n}}}$ haben wir folgendes: 
    
    \begin{center}
    $|w_{n}| = 2^{2^{2n}}$ \\
    $\log_{2}(|w_{n}|) = 2^{2n}$ \\
    $\log_{2} \log_{2} |w_{n}|= 2n$ \\
    $\frac{1}{2} log_{2} \log_{2} |w_{n}|= n$
    \end{center}
    
    \par
	
	$\Rightarrow K(w_{n}) \le \log_{2}(\frac{1}{2} \log_{2} \log_{2} |w_{n}|) + c$ \\
	$\Rightarrow K(w_{n}) \le \log_{2}(\log_{2} \log_{2} |w_{n}|) - \log_{2}(2) + c$ \\
	$\Rightarrow K(w_{n}) \le \log_{2}(\log_{2} \log_{2} |w_{n}|) + c'$ wobei c' = c - $\log_{2}(2)$
	
	\section*{Aufgabe 4b}
	
	$K(a_{n}) \le \log_{2} \log_{2} \sqrt[3]{|a_{n}|} + c$ \\
	$\Rightarrow n =  \log_{2} \sqrt[3]{|a_{n}|}$ \\
	$\Leftrightarrow n = \frac{1}{3} \log_{2} |a_{n}| $ \\
	$\Leftrightarrow 3n = \log_{2} |a_{n}| $ \\
	$\Leftrightarrow 2^{3n} = |a_{n}| $
	
	 
	
	Von der Definition 2.18 des Buches wissen wir dass $K(y_{n}) = K(Bin(y_{n}))$ gilt somit definieren wir ein $a_{n} = Bin(y_{n}) = 1(0)^{2^{3n} - 1}$
    
    Wir definieren die Zahlenfolge $y_{1}, y_{2}, y_{3}...$ durch $y_{n} = 2^{3n}$ für alle $ n \in \mathbb{N} - {0}$.
    Offenbar gilt $y_{n} < y_{n+1}$ und $n = \log_{2} \sqrt[3]{y_{n}}$
    
    \section*{Aufgabe 6}
    
    Wir definieren $A_{n}$, $\forall n \in \mathbb{N}$ als die Menge von allen natürlichen in dem Intervall:
    \begin{center}
    $[2^n, 2^{n+1} - 1]$, bzw. $A_{n} = \{w_{n}|$ $2^n \le w_{n} \le 2^{n+1} - 1, w_{n} \in \mathbb{N}\}$
    \end{center}
    
    
    \par 
    \noindent
    Wir wissen von der Aussage, dass die untere Schranke von jeder Menge $A_{n}$: 
	\begin{center}
	$2^n$ ist. \\
\end{center}	    
   
   	\noindent
    Jetzt möchten wir alle natürlichen Zahlen von diesen Mengen $A_{n}$ als binäre Kodierung repräsentieren. \\
    Die äquivalenten Mengen für die binäre Kodierung von den natürlichen Zahlen der Mengen $A_{n}$ sind:
	\begin{center}
	 $B_{n} = (\Sigma_{bool})^n $, $\forall n \in \mathbb{N}$
\end{center}	    
    
    
    \par 
  	\noindent
  	Von dieser Äquivalenz, definieren wir die folgende Gleichung: 
	\begin{center}
	$Bin(w_{n}) = y_{n}$ wobei $w_{n} \in A_{n}$ und $y_{n} \in B_{n}$, $\forall n \in \mathbb{N}$
\end{center}	  	
  	
  	\par 
  	\noindent
  	Ausserdem, alle Wörter $y_{n} \in B_{n}$ haben eine Länge von $n + 1$ bzw. $|y_{n}| = n + 1$
  	
    \par 
    \noindent
    Somit haben wir eine Kolmogorov-Komplexität von: 
    \begin{center}
    $|yn| + c \ge K(y_{n})$
    \end{center}
\end{document}
